\documentclass[12pt, letterpaper]{article}
\usepackage[letterpaper, total={6.5in,9in}]{geometry}
\usepackage{graphicx}
\usepackage{amsmath}
\usepackage{booktabs}
\usepackage{indentfirst}
\usepackage[colorlinks=true,
linkcolor=red]{hyperref}

\title{Lab 02 -- Hooke's Law \& Simple Harmonic Motion}
\author{Colin Lambert, Erik Ahl, Ethaniel Sianipar}
\date{2026-01-28}

\begin{document}
\maketitle

\section{Results}
We designed two different experiments to measure the $k$ value a given spring.
Both experiments made use of the \begin{equation}
    F_s = -k\cdot x
    \label{eq:hookes}
\end{equation}
where $F_s$ is the spring force acting on an object, $k$ is the spring constant, and $x$ is the displacement from the equilibrium point of the spring.
In our first experiment, we used five different masses and statically analyzed the system to find a $k$ value of $9.49 \pm 0.1 \left[\frac{N}{m}\right]$. (see \hyperref[fig:data-1]{Figure \ref{fig:data-1}})
In our second experiment, we dynamically analyzed the system while it was is simple harmonic motion (SHM). We found the $k = 9.3 \pm 0.1 \left[\frac{N}{m}\right]$. (see \hyperref[fig:data-2]{Figure \ref{fig:data-2}})

While each of the measurements were equally uncertain, we decided that the measurement using both of the Lab Quest sensors was more precise, and had less variability than Experiment 1 did. We also decided the system in Experiment 2 was able to stay more stable and constant, while everything done in Experiment had some variability from things shaking due to the movement of the lab group.

\begin{figure}
    \centering
    \includegraphics[width=0.75\linewidth]{RulerTrial.png}
    \caption{The relationship between the force (F) of a mass on a spring and the position (x) of that mass. On this graph there is 4 masses used and the data records it's displacement from the equilibrium of each masses. The line of best fit is graphed as well and is represented by the equation $F= 9.5x \pm 0.1 \text{ N/m}$. The slope of the line of best fit represents the force constant of the spring.
    \label{fig:placeholder}
\end{figure}



\begin{figure}
    \centering
    \includegraphics[width=0.75\linewidth]{figure2a.png}
    \caption{The relationship between the force (F) of a mass on a spring and the position (x) of that mass while it was suspended from a spring and oscillating vertically. The mass used weighed         110 grams and the data was collected over 1.68 seconds. The line of best fit is graphed as well and is represented by the equation $F = -9.3x \pm 0.1 \text{ N/m}$. The slope of the line of          best fit represents the force constant of the spring.}
    \label{fig:placeholder}
\end{figure}

\end{document}
